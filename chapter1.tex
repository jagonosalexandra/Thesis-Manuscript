\chapter{INTRODUCTION}

{\baselineskip=2\baselineskip
\section{Background of the Study}
Eggplant (\textit{Solanum melongena} L.) has been cultivated for centuries and is now a staple in cuisines worldwide. Also known as aubergine, brinjal, or \textit{talong} in the Philippines, eggplant holds a prominent position in the vegetable sector, with global production of approximately 60.8 million metric tons from an area of over 1.9 million hectares \citep{FAOSTAT_EggplantProduction}. In April to June of 2023, the Philippines produced 102.98 thousand metric tons of eggplant \citep{PSA_Eggplant_2023}, making it one of the most important and widely grown vegetable crops in the country.

But before being sold, eggplant is manually categorized by visual inspection to determine their quality \citep{sun2025novel}. This includes assessing the fruit's diameter, length, volume, curvature, color homogeneity, calyx area (the green, leaf-like “cap” at the fruit's stem end), calyx color, and surface defects \citep{lyu2025agronomic,lalam2025automatic}. Sorting only takes a few hours for small-scale farmers, but it can take a little longer and more workers on large plantations, since the quantity of harvested eggplants is directly proportional to labor and time needed \citep{Khan2025UsingCV}.

Because most people just use their eyes to judge the quality of eggplant, there is a significant rate of inaccuracy in addition to the time-consuming and expensive nature of manual eggplant sorting \citep{Waghmare2025Comprehensive}.  As a result, a number of cutting-edge technologies were combined to increase classification accuracy and decrease inefficiencies, particularly deep learning and computer vision.  For instance, the study of


%-----------------------------------------------------------------------------------------------

%-----------------------------------------------------------------------------------------------------------------------------------
\section{Statement of the Problem}

This study seeks to investigate some properties of decomposable hyper KS-semigroups in the context of strong, weak, quasi- and bi-hyper KS-ideals.

\section{Objectives of the Study}

In view of the above stated problem, we have the following objectives:
\begin{enumerate}
	\item To introduce the concept of strong, weak, quasi- and bi-hyper KS-ideals;
	\item To provide characterizations of strong, weak, quasi- and bi-hyper KS-ideals and investigate their relationships;
	\item To introduce the idea of decomposable hyper KS-semigroups and give some characterizations.
\end{enumerate}

\section{Significance of the Study}

The concept of hyperstructures is itself, a powerful mathematical tool since algebraic hyperstructures seem to occur very naturally in many areas of mathematics and even in other disciplines. 

\section{Scope and Limitations}

The primary motivation of this study lies within the structural properties of hyper 

\section{Definition of Terms}

\begin{description}

	\item[Data Logger] 
	An electronic device that records data over time or in relation to location either with a built-in instrument or sensor or via external instruments and sensors.
	
	\item[GPS Tracking] 
	Using the Global Positioning System to determine and track the precise location of a person, vehicle, or other asset.
	
	\item[Real-time Monitoring] 
	The process of continuously observing a system or process and immediately reporting any changes or anomalies.
	
	\item[Sensor] 
	A device that detects or measures a physical property and records, indicates, or otherwise responds to it.
	
	\item[Telemetry] 
	The process of recording and transmitting the readings of an instrument.
	
	\item[Wireless Communication] 
	The transfer of information between two or more points that are not connected by an electrical conductor.

\end{description}

}
